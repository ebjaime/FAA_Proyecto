\documentclass{esannV2}
\usepackage{graphicx}
\usepackage[utf8]{inputenc}
\usepackage{amssymb,amsmath,array}

\usepackage{hyperref}
\usepackage{listings}
\lstset{
  language=bash,
  basicstyle=\ttfamily
}
%***********************************************************************
% !!!! IMPORTANT NOTICE ON TEXT MARGINS !!!!!
%***********************************************************************
%
% Please avoid using DVI2PDF or PS2PDF converters: some undesired
% shifting/scaling may occur when using these programs
% It is strongly recommended to use the DVIPS converters, and to submit
% PS file. You may submit a PDF file if and only if you use ADOBE ACROBAT
% to convert your PS file to PDF.
%
% Check that you have set the paper size to A4 (and NOT to letter) in your
% dvi2ps converter, in Adobe Acrobat if you use it, and in any printer driver
% that you could use.  You also have to disable the 'scale to fit paper' option
% of your printer driver.
%
% In any case, please check carefully that the final size of the top and
% bottom margins is 5.2 cm and of the left and right margins is 4.4 cm.
% It is your responsibility to verify this important requirement.  If these margin requirements and not fulfilled at the end of your file generation process, please use the following commands to correct them.  Otherwise, please do not modify these commands.
%
\voffset 0 cm \hoffset 0 cm \addtolength{\textwidth}{0cm}
\addtolength{\textheight}{0cm}\addtolength{\leftmargin}{0cm}

%***********************************************************************
% !!!! USE OF THE esannV2 LaTeX STYLE FILE !!!!!
%***********************************************************************
%
% Some commands are inserted in the following .tex example file.  Therefore to
% set up your ESANN submission, please use this file and modify it to insert
% your text, rather than staring from a blank .tex file.  In this way, you will
% have the commands inserted in the right place.

\begin{document}
%style file for ESANN manuscripts
\title{El título que queráis}

%***********************************************************************
% AUTHORS INFORMATION AREA
%***********************************************************************
\author{Vuestros nombres$^1$
%
% Optional short acknowledgment: remove next line if non-needed
%\thanks{This is an optional funding source acknowledgement.}
%
% DO NOT MODIFY THE FOLLOWING '\vspace' ARGUMENT
\vspace{.3cm}\\
%
% Addresses and institutions (remove "1- " in case of a single institution)
%1- School of First Author - Dept of First Author \\
%Address of First Author's school - Country of First Author's
%school
%
% Remove the next three lines in case of a single institution
%\vspace{.1cm}\\
%2- School of Second Author - Dept of Second Author \\
%Address of Second Author's school - Country of Second Author's school\\
}
%***********************************************************************
% END OF AUTHORS INFORMATION AREA
%***********************************************************************

\maketitle

\begin{abstract}
Resumen del trabajo realizado en 100 palabras.
\end{abstract}

\section{Introducción}
El trabajo del proyecto se debe presentar usando esta plantilla que se puede
obtener para latex y word en el siguiente enlace

\begin{center}
\href{https://www.elen.ucl.ac.be/esann/index.php?pg=guidelines}{https://www.elen.ucl.ac.be/esann/index.php?pg=guidelines}
\end{center}
 
No se pueden cambiar las especificaciones de la plantilla: márgenes, tipo de
letra, etc. Se recomienda usar latex.

\section{Organización de la entrega}
El proyecto se organiza en las siguientes actividades con su correspondiente
peso en la nota del proyecto:

\begin{itemize}
\item Presentación breve de tema de aprendizaje automático [10\%]
\item Memoria sobre el trabajo realizado [60\%]
\item Presentación final del trabajo el día de examen final de enero [30\%]
\end{itemize}

La memoria y la presentación en formato ppt o pdf se puede entregar hasta el día
antes de la fecha de la presentación.

\section{Memoria}
La longitud máxima serán \emph{8 caras} sin incluir referencias. Se valorará la
capacidad de síntesis por lo que superar las 8 páginas tendrá penalización. La
memoria se deben tratar, de forma orientativa, los siguientes aspectos:

\begin{itemize}
\item Introducción [1pt]: breve introducción al problema a analizar, descripción del dataset y objetivos.
\item Análisis exploratorio de los datos [1pt]: Descripción estadística de los datos: Número de clases, distribución de las clases, otras estadísticas y análisis.
\item Descripción de los distintos atributos  propuestos y cómo se obtienen [2pt]
Modelos utilizados, descripción del protocolo experimental, estimación de parámetros, etc [2pt]: En esta sección se debe especificar toda la información necesaria para que otra persona, sin acceso a vuestro código, pueda reproducir los experimentos que habéis hecho. Debe quedar claro en la descripción que no se usan los datos de test para entrenar los modelos.
\item Resultados obtenidos en forma tabular y/o usando gráficas [1pt]. Se debe describir que muestra cada tabla o gráfica.
\item Discusión de los resultados obtenidos y conclusiones [2pt] Esta sección es la más importante del documento ya que es dónde se pone en valor el trabajo realizado. Debéis responder a preguntas  tipo ¿Qué atributos y métodos han dado mejores resultados? ¿Por qué creéis que es así? ¿Son resultados aceptables? ¿Qué modelos recomendaríais bajo qué condiciones? Tal vez un modelo funcione mejor cuando se entrena con pocos datos o funcione mejor para clasificar una de las clases y peor para otras, etc.
\item Además se deben utilizar al menos dos de las técnicas descritas a los largo del curso por vuestros compañeros [1pt]
\end{itemize}

Se valorará la correcta redacción del documento.

\section{Presentación}
Debéis entregar un ppt o pdf con el resumen del trabajo. Debe ser una
presentación para presentar en 12 minutos. El tiempo de presentación será
estricto y se parará a los que se pasen de tiempo.

\section{Tablas y figuras}
Las tablas y figuras hay que referenciarlas desde el texto y describir qué
muestran. Por ejemplo, en la Figura~\ref{fig:logo} se muestra el logo de
scikit-learn. En la Tabla~\ref{tab:ageweight} se muestra un ejemplo de tabla.


\begin{figure}[b!]
\centering
\includegraphics{logo.png}
\caption{Logo de scikit-learn}\label{fig:logo}
\end{figure}

\begin{table}[t!]
  \centering
  \begin{tabular}{|c|c|c|}
    \hline
    ID & age & weight \\
    \hline
    1& 15 & 65 \\
    2& 24 & 74\\
    3& 18 & 69 \\
    4& 32 & 78 \\
    \hline
  \end{tabular}
  \caption{Age and weight of people.}\label{tab:ageweight}
\end{table}

\section{Citas}
Es una buena práctica referenciar los trabajos en los que se ha basado vuestro
análisis. Por ejemplo, si los experimentos se han realizado utilizado
scikit-learn habría que citarlo~\cite{scikit-learn}.

\section{Cómo obtener este pdf usando \LaTeX}
Podéis seguir los siguientes pasos para obtener el pdf junto con sus
referencias desde una consola linux:

\begin{lstlisting}
 pdflatex fuente.tex
 bibtex fuente
 pdflatex fuente.tex
 pdflatex fuente.tex
\end{lstlisting}

Estos comandos generan el fichero pdf que incluye las referencias. Las
referencias se guardan en un fichero aparte (en esta caso mi\_bibliografia.bib).
% ****************************************************************************
% BIBLIOGRAPHY AREA
% ****************************************************************************

\begin{footnotesize}

% IF YOU DO NOT USE BIBTEX, USE THE FOLLOWING SAMPLE SCHEME FOR THE REFERENCES
% ----------------------------------------------------------------------------
%\begin{thebibliography}{99}
%
% For books
%\bibitem{Haykin_book} S. Haykin, editor. \emph{Unsupervised Adaptive Filtering vol.1 : Blind Source Separation}, John Willey ans Sons, New York, 2000.
%
% For articles
%\bibitem{DelfosseLoubaton_article}N. Delfosse and P. Loubaton, Adaptibe blind separation of sources: A deflation
%approach, \emph{Signal Processing}, 45:59-83, Elsevier, 1995.
%
%% For paper in proceedings published as serie books (LNCS,...)
%\bibitem{CrucCichAmari_bookproceedings} S. Cruces, A. Cichocki and S. Amari, The minimum entropy and cumulants based contrast functions for blind source extraction. In J. Mira and A. Prieto, editors, proceedings of the 6$^{th}$ \emph{international workshop on artificial neural networks} ({IWANN} 2001), Lecture Notes in Computer Science 2085, pages 786-793,
%Springer-Verlag, 2001.
%
%% For paper in conference proceedings
%\bibitem{VrinsArchambeau_proceedings} F. Vrins, C. Archambeau and M. Verleysen, Towards a local separation performances estimator using common ICA contrast functions? In M. Verleysen, editor, \emph{proceedings of the $12^{th}$
%European Symposium on Artificial Neural Networks} ({ESANN} 2004),
%d-side pub., pages 211-216, April 28-30, Bruges (Belgium), 2004.
%
%% For Technical Report
%\bibitem{Stone_TechRep} J. V. Stone and J. Porrill, Undercomplete independent component analysis for signal separation and dimension
%reduction. Technical Report, Psychology Department, Sheffield
%University, Sheffield, S10 2UR, England, October 1997.
%\end{thebibliography}
% ----------------------------------------------------------------------------

% IF YOU USE BIBTEX,
% - DELETE THE TEXT BETWEEN THE TWO ABOVE DASHED LINES
% - UNCOMMENT THE NEXT TWO LINES AND REPLACE 'Name_Of_Your_BibFile'

\bibliographystyle{unsrt}
\bibliography{mi_bibliografia}

\end{footnotesize}

% ****************************************************************************
% END OF BIBLIOGRAPHY AREA
% ****************************************************************************

\end{document}
